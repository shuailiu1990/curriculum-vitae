%%%%%%%%%%%%%%%%%%%%%%%%%%%%%%%%%%%%%%%%%
% Plain Cover Letter
% LaTeX Template
% Version 1.0 (28/5/13)
%
% This template has been downloaded from:
% http://www.LaTeXTemplates.com
%
% Original author:
% Rensselaer Polytechnic Institute 
% http://www.rpi.edu/dept/arc/training/latex/resumes/
%
% License:
% CC BY-NC-SA 3.0 (http://creativecommons.org/licenses/by-nc-sa/3.0/)
%
%%%%%%%%%%%%%%%%%%%%%%%%%%%%%%%%%%%%%%%%%

%----------------------------------------------------------------------------------------
%	PACKAGES AND OTHER DOCUMENT CONFIGURATIONS
%----------------------------------------------------------------------------------------

\documentclass[11pt,a4paper,sans]{letter} % Default font size of the document, change to 10pt to fit more text
\usepackage{geometry}
\usepackage{enumitem}
\geometry{top=6.0cm, bottom=0cm}
%\usepackage{newcent} % Default font is the New Century Schoolbook PostScript font 
%\usepackage{helvet} % Uncomment this (while commenting the above line) to use the Helvetica font

% Margins
\topmargin=-1.2in % Moves the top of the document 1 inch above the default
\textheight=10in%8.5in % Total height of the text on the page before text goes on to the next page, this can be increased in a longer letter
\oddsidemargin=-5pt % -10pt % Position of the left margin, can be negative or positive if you want more or less room
\textwidth=6.5in % Total width of the text, increase this if the left margin was decreased and vice-versa
\date{}
\let\raggedleft\raggedright % Pushes the date (at the top) to the left, comment this line to have the date on the right

\begin{document}

%----------------------------------------------------------------------------------------
%	ADDRESSEE SECTION
%----------------------------------------------------------------------------------------

\begin{letter}{16th December 2022 \\
University of Geneva-Gravitational-Wave Science Center\\
Geneva 1290\\ 
Geneva, Switzerland}

Shuai Liu \\
TianQin Research Center for Gravitational Physics \\
Zhuhai Campus Sun Yat-sen University \\
Tangjiawan, Zhuhai 519082 \\
Guangdong, People's Republic of China \\
shuai.liu.1990@outlook.com




%----------------------------------------------------------------------------------------
%	YOUR NAME & ADDRESS SECTION
%----------------------------------------------------------------------------------------

%\begin{center}
%\large\bf Dr. John Smith \\ % Your name
%%\vspace{20pt} \hrule height 1pt % If you would like a horizontal line separating the name from the address, uncomment the line to the left of this text
%123 Broadway \\ City, State 12345 \\ (000) 111-1111 % Your address and phone number
%\end{center} 
%\vfill

%\signature{John Smith} % Your name for the signature at the bottom

%----------------------------------------------------------------------------------------
%	LETTER CONTENT SECTION
%----------------------------------------------------------------------------------------

\opening{To whom it may concern,} 
 
    I am writing to express my sincere interest in postdoctoral positions on the physics of gravitational-wave-source
    progenitors and population modelling of stellar-origin GW sources for the next-generation GW detectors, such as the
    ET and LISA. I would love to
pursue my postdoctoral research in Gravitational-Wave Science Center, University of Geneva. I am a postdoctoral fellow, who is working on 
   stellar-origin binary black holes with gravitational waves at Sun Yat-sen University.

I have been working on exploring binary black holes with gravitational waves, since I was a PhD
    candidate. During my PhD study, I focused on exploring stellar-mass binary black holes by future
    space-borne gravitational wave detectors (e.g., LISA and TianQin). Specifically, we estimated the detection
    capabilities of
    these detectors, e.g., detection number and measurement precision for source parameters. We also explored the
    potential to distinguish the formation channels by measuring orbital eccentricities of sources with future detectors
    (e.g., LISA, ET). During my postdoctoral
    stage, we studied the GW190521-like stellar-mass binary black holes whose primary mass and merger remnant are in the
    mass gap and mass range of 
    intermediate-mass black holes, respectively. We explored this special kind of stellar-mass binary black holes as
    before, including investigating the potential to distinguish the formation
    channels by measuring orbital eccentricities of sources. 
    Meanwhile, I participated
    in a project which inferred the Hubble
    constant with gravitational wave signals from stellar-mass binary black holes as one of the most important
    contributors. 
    Now, I am working on
    hierarchical triple
    black hole systems whose inner binaries are stellar-mass binary black holes in Population III star clusters. I plan to
    study the population properties (e.g., mass and orbital eccentricity) of these binary black holes in the triple systems by future gravitational wave
    detections (e.g., LISA and ET), and then
    infer their formation channels depending on these results. I have completed the simulation for Population III
    clusters by $N$-body and population synthesis codes and selected target hierarchical triple systems.  
    In the future, I am going to study hierarchical triple black holes systems in globular clusters, of which inner binary black holes are
    stellar-mass black holes and tertiaries are intermediate-mass black holes. The gravitational waves from inner binary
    black holes would bear information of intermediate-mass black holes, because of the gravitational perturbation from the
    latter on the former. I will infer the existences of
    intermediate-mass black holes with stellar-mass binary black holes detected by future gravitational waves detectors.
    This research will provide a novel approach to detect intermediate-mass black holes.

I am a self-motivated, independent researcher with
    passions for exploring stellar- and intermediate-mass black holes with future detectors (e.g., LISA and ET). I think
    my research is closely relative to the research field in your group. I believe that my knowledge,
    technical skills and experience (e.g., gravitational waves, $N$-body and population synthesis simulation) in studying binary black holes by
    gravitational waves could benefit and enrich scientific payoff
from each other. 

         
Sincerely,

Shuai Liu


%\encl{Curriculum vitae, employment form} % List your enclosed documents here, comment this out to get rid of the "encl:"

%----------------------------------------------------------------------------------------

\end{letter}

\end{document}
