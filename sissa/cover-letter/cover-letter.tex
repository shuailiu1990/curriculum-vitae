%%%%%%%%%%%%%%%%%%%%%%%%%%%%%%%%%%%%%%%%%
% Plain Cover Letter
% LaTeX Template
% Version 1.0 (28/5/13)
%
% This template has been downloaded from:
% http://www.LaTeXTemplates.com
%
% Original author:
% Rensselaer Polytechnic Institute 
% http://www.rpi.edu/dept/arc/training/latex/resumes/
%
% License:
% CC BY-NC-SA 3.0 (http://creativecommons.org/licenses/by-nc-sa/3.0/)
%
%%%%%%%%%%%%%%%%%%%%%%%%%%%%%%%%%%%%%%%%%

%----------------------------------------------------------------------------------------
%	PACKAGES AND OTHER DOCUMENT CONFIGURATIONS
%----------------------------------------------------------------------------------------

\documentclass[11pt,a4paper,sans]{letter} % Default font size of the document, change to 10pt to fit more text
\usepackage{geometry}
\geometry{top=6.0cm, bottom=0cm}
%\usepackage{newcent} % Default font is the New Century Schoolbook PostScript font 
%\usepackage{helvet} % Uncomment this (while commenting the above line) to use the Helvetica font

% Margins
\topmargin=-1in % Moves the top of the document 1 inch above the default
\textheight=10in%8.5in % Total height of the text on the page before text goes on to the next page, this can be increased in a longer letter
\oddsidemargin=-5pt % -10pt % Position of the left margin, can be negative or positive if you want more or less room
\textwidth=6.5in % Total width of the text, increase this if the left margin was decreased and vice-versa
\date{}
\let\raggedleft\raggedright % Pushes the date (at the top) to the left, comment this line to have the date on the right

\begin{document}

%----------------------------------------------------------------------------------------
%	ADDRESSEE SECTION
%----------------------------------------------------------------------------------------

\begin{letter}{March 20 2022 \\
%Prof. Alvin Chua and Prof. Yanbei Chen \\
SISSA \\
Via Bonomea, 265 \\ 
Trieste, 34136, Italy}

Shuai Liu \\
TianQin Research Center for Gravitational Physics \\
Zhuhai Campus Sun Yat-sen University \\
Tangjiawan, Zhuhai 519082 \\
Guangdong, People's Republic of China \\
shuai.liu.1990@outlook.com




%----------------------------------------------------------------------------------------
%	YOUR NAME & ADDRESS SECTION
%----------------------------------------------------------------------------------------

%\begin{center}
%\large\bf Dr. John Smith \\ % Your name
%%\vspace{20pt} \hrule height 1pt % If you would like a horizontal line separating the name from the address, uncomment the line to the left of this text
%123 Broadway \\ City, State 12345 \\ (000) 111-1111 % Your address and phone number
%\end{center} 
%\vfill

%\signature{John Smith} % Your name for the signature at the bottom

%----------------------------------------------------------------------------------------
%	LETTER CONTENT SECTION
%----------------------------------------------------------------------------------------

\opening{To whom it may concern,} 
 
I am writing to express my sincere interest in postdoctoral positions in the astroparticle physics and theoretical
    and scientific data science group at SISSA. I would love to
pursue my postdoctoral studies as a member of your team. I am a postdoctoral fellow, who is working on black holes with
    gravitational waves at Sun Yat-sen University.

I have been working on gravitational wave astrophysics, especially binary black holes, since I was a PhD candidate. I
    have rich astrophysics knowledge on binary black holes and rich technical skills (e.g., Bayesian inference and
    $N$-body simulation) to investigate them by
    gravitational waves. During my PhD study, I focused on exploring stellar-mass binary black holes by future
    space-borne gravitational wave detectors, i.e., LISA and TianQin. Specifically, I and collaborators estimated detection number and
    measurement precision for source parameters with Fisher information matrix. During my postdoctoral stage, I shifted
    my attention to intermediate-mass black holes. I and collaborators assessed the detection capacity of LISA and TianQin
    for GW190521-like binary black holes whose merger remnants are intermediate-mass black holes, and constrained the
    Hubble constant by the Bayesian method. I also participated in performing the Bayesian method on constraining the Hubble
    constant by stellar-mass binary black holes as one of the most important contributors. Currently, I am working on
    hierarchical triple
    black holes systems with intermediate-mass black holes in Population III star clusters. I have completed simulation for
    clusters by $N$-body codes, and I am working on detection for the triple systems with gravitational detectors.  
    In the future, I plan to perform hierarchical Bayesian inference on constraining the formation and evolution of massive binary black holes by gravitational waves,
    which also involves semi-analytical simulation and source parameter estimation with the Bayesian method. In the processes of
    completing the above works, I can achieve the goals by writing codes or modifying existing codes and optimize codes
    quickly. In addition, I am also interested in improving execution efficiency of codes and learning new programming
    languages. 

I would appreciate the opportunity to work in your group at SISSA. I am a self-motivated, independent researcher with
    passions for science as well as interests for codes. I believe that the nature of my research, my knowledge,
    technical skills and experience (e.g., Bayesian method and $N$-body simulation) on studying black holes by
    gravitational waves are relative to your expectations and could aid in your research. I have attached my CV and would be very interested in setting up a time further
discuss my skills and qualifications with you. Please let me know whether you have questions and I look forward to
hearing from you.\\

Sincerely,

Shuai Liu


%\encl{Curriculum vitae, employment form} % List your enclosed documents here, comment this out to get rid of the "encl:"

%----------------------------------------------------------------------------------------

\end{letter}

\end{document}
