%%%%%%%%%%%%%%%%%%%%%%%%%%%%%%%%%%%%%%%%%
% "ModernCV" CV and Cover Letter
% LaTeX Template
% Version 1.1 (9/12/12)
%
% This template has been downloaded from:
% http://www.LaTeXTemplates.com
%
% Original author:
% Xavier Danaux (xdanaux@gmail.com)
%
% License:
% CC BY-NC-SA 3.0 (http://creativecommons.org/licenses/by-nc-sa/3.0/)
%
% Important note:
% This template requires the moderncv.cls and .sty files to be in the same
% directory as this .tex file. These files provide the resume style and themes
% used for structuring the document.
%
%%%%%%%%%%%%%%%%%%%%%%%%%%%%%%%%%%%%%%%%%

%----------------------------------------------------------------------------------------
%	PACKAGES AND OTHER DOCUMENT CONFIGURATIONS
%----------------------------------------------------------------------------------------

\documentclass[12pt,a4paper,sans]{article}%{moderncv} % Font sizes: 10, 11, or 12; paper sizes: a4paper, letterpaper, a5paper, legalpaper, executivepaper or landscape; font families: sans or roman
\usepackage{setspace} 
\setstretch{1.2}

%\moderncvstyle{classic} % CV theme - options include: 'casual' (default), 'classic', 'oldstyle' and 'banking'
%\moderncvcolor{grey} % CV color - options include: 'blue' (default), 'orange', 'green', 'red', 'purple', 'grey' and 'black'

\usepackage{lipsum} % Used for inserting dummy 'Lorem ipsum' text into the template
%\usepackage{caption}
%\captionsetup{font=10}
\usepackage[scale=0.75]{geometry} % Reduce document margins
%\setlength{\hintscolumnwidth}{3cm} % Uncomment to change the width of the dates column
%\setlength{\makecvtitlenamewidth}{10cm} % For the 'classic' style, uncomment to adjust the width of the space allocated to your name

%----------------------------------------------------------------------------------------
%	NAME AND CONTACT INFORMATION SECTION
%----------------------------------------------------------------------------------------
%\title{Research Statement}
%\maketitle
%\firstname{Shuai} % Your first name
%\familyname{Liu} % Your last name

% All information in this block is optional, comment out any lines you don't need
\title{\vspace{-2.5cm}\Huge Research Statement \vspace{-2.2em}}
\date{}
%\section{Research Statement}
%\address{W. Ethan Eagle}{}
%\mobile{(302) 584 3464}
%\phone{(000) 111 1112}
%\fax{(000) 111 1113}
%\email{}
%\homepage{staff.org.edu/~jsmith}{staff.org.edu/$\sim$jsmith} % The first argument is the url for the clickable link, the second argument is the url displayed in the template - this allows special characters to be displayed such as the tilde in this example
%\extrainfo{additional information}
%\photo[70pt][0.4pt]{pictures/picture} % The first bracket is the picture height, the second is the thickness of the frame around the picture (0pt for no frame)
%\quote{"A witty and playful quotation" - John Smith}

%----------------------------------------------------------------------------------------

\begin{document}
\maketitle
%\makecvtitle % Print the CV title

%----------------------------------------------------------------------------------------
My research interests focus on exploring promising sources for future gravitational wave detectors, for example, stellar- and
intermediate-mass black holes as well as massive black holes. Specially, my research is to assess detection capacity of
these detectors for black holes, and to study their formation and evolution. My research statement is organized as
follows:  

I worked on stellar-mass binary black holes, during my PhD study. Stellar-mass black holes detected by
LIGO/Virgo are heavier than those observed by electromagnetic waves, so they are promising sources for scheduled
space-borne gravitational waves detectors focusing on milihertz, e.g., LISA and TianQin. We adopted five mass
distribution models of stellar-mass binary balck holes calibarted by events detected to assess detection capacity of
LISA and TianQin: dozens of sources could be resolved, source parameters could be measured accurately, and formation
scenarios could be distinguished by measuring orbital eccentricities. We also investigated the improvement for detection capacity by joint
observation of TianQin and LISA. This work [published in ] laid a foundation for future data analysis and provides
guidance for joint observation of detector networks. 

I focus on intermediate-mass black holes, after I become a postdoctral fellow. LIGO/Virgo observed the stellar-mass binary black hole named
GW190521 whose merger remnant falls in mass range of intermediate-mass black holes. The event directly confirms the existence of
intermediate-mass black hole and that intermediate-mass black hole could be formed by merger of stellar-mass binary
black holes for the first time. We studied GW190521-like binary black holes by TianQin and LISA. A dozen of this kind of
source could be detected and their source parameters could be measured with high precision. We discussed probability to identify GW190521-like binary black holes with extremely large orbital eccentricities by
multiband observation of future space-borne detectors and the third generation ground-based detectors. We also performed the Bayesian method
on constraining the Hubble constant (10\% precision) with this kind of source. This work [published in ] provides
gauidence for detecting GW190521-like by detectors and studying cosmology in the future. Meanwhile, I participated in constraining the Hubble
constant with stellar-mass binary black holes detected by muliband observation as one of important contributors. We
adopted the Bayesian method to infer the Hubble constant, it could be constrained with about 1\%. This work is of great
importance for the study of cosmology with stellar-mass black holes detected by future multiband observation. Now, I am working on
detecting intermediate-mass black holes in Population III star clusters by furture detectors. Intermediate-mass black
holes could be formed by massive stars with low metcalicity and form hierachical triple black holes with stellar-mass
black holes. I plan to study intermediate-mass black holes by stellar-mass black hole around them detected by gravitational waves in three apects: distribution of orbital
eccentricities, evoluiton of orbital eccentricities and multiband observation number. I have completed evolution of
Population III star clusters by $N$-body simulation and selected triple systems. I am working on the detection for 
binary stellar-mass black holes in triple systems with gravitational waves. This work will provide a framework to noval
approach to detect and study intermediate-mass black holes with gravitational waves.

In the future, I am going to constrain the formation and evolution of massive black holes with future gravitational
wave detector networks. This project is relative your.   

Intermediate-mass black holes (IMBHs) are very important for astronomy and physics, because they may shed
light on the formation of massive black holes, the evolution of galaxies, as well as the effect of dark matter. However, IMBHs
are still elusive, due to the lack of direct observation for them. In the case that the detection for IMBHs with
electromagnetic waves is challenging, gravitational waves are expected to contribute significant help for studying them 
as an entirely new probe. So, my research concentrates on the detection for IMBHs with gravitational waves.

The IMBHs could be sought by observing GW190521-like binary black holes, due to the
remnant of GW190521 with $\sim163M_{\odot}$. This kind of binary black holes would be in the sensitive bands of the
future space-borne gravitational wave detectors, such as LISA and TianQin, and last for several years before they
merge into LIGO/Virgo bands. So, in my past about one year as a postdoctoral researcher, I simulated this kind of binary
black hole in the Universe depending on the masses and merger rate of GW190521. I studied the capacity for detection of GW190521-like
binary black holes with LISA/TianQin and the potential to constrain their formation scenarios by measuring the orbital
eccentricities. As the number of this kind of binary black holes detected increases, more accurate population models will
be released. I will use the updated population model to obtain more accurate predictions of space gravitational wave
detectors. IMBHs with more masses could be formed in globular
clusters, and they would form a hierarchical triple system with stellar-mass binary black holes around them. The IMBHs
will affect the evolution of orbital elements of stellar-mass binary black holes in these systems, making the
gravitational waves from them carry the information on IMBHs. The stellar-mass binary black holes during the above phase
are far away from the merger, they fall in the LISA/TianQin bands. So the hierarchical triple system could be used to
detect the IMBHs in globular clusters, which is the research I am working on. Specially, I simulate the evolution of
globular clusters hosting IMBHs by population syntheses and N-body simulations and select the hierarchical triple
systems. I calculate the waveform from the stellar-mass binary black holes and will study the existence and masses of
IMBHs with the population properties of sources resolved by LISA/TianQin. For the sources undetectable, I plan to use
them to study IMBHs by the multiband observation with LIGO/Virgo/KAGRA. This research will provide a novel approach to
detecting IMBHs.

%The IMBHs could be sought by observing GW190521-like binary black holes, due to the
%remnant of GW190521 with $\sim163M_{\odot}$. This kind of binary black holes would be in the sensitive bands of the
%future space-borne gravitational wave detectors, such as LISA and TianQin, and last for serveral years before they
%merge into LIGO/Virgo bands. So, in my past about one year as a postdoctoral researcher, I studied the capacity for detection of GW190521-like
%binary black holes with LISA/TianQin. I simulated this kind of binary black hole in the Universe depending on the
%masses and merger rate of GW190521, and then calculated the expected detection number and source parameter estimation
%precision by the Fisher information matrix. The results showed that LISA/TianQin would be able to tell LIGO/Virgo/KAGRA when and where they would merge to IMBHs in advance, depending on the high estimation precision on merger time and sky localization. In addition, the formation scenarios and sites of
%these sources could be constrained with the orbital eccentricities measurement. IMBHs with more masses could be formed in globular
%clusters, and they would form a hierarchical triple system with stellar-mass binary black holes around them. The IMBHs
%will affect the evolution of orbital elements of stellar-mass binary black holes in these systems, making the
%gravitational waves from them carry the information on IMBHs. The stellar-mass binary black holes during the above phase
%are far away from the merger, they fall in the LISA/TianQin bands. So the hierarchical triple system could be used to
%detect the IMBHs in globular clusters, which is the research I am working on. Specially, I simulate the evolution of
%globular clusters hosting IMBHs by population syntheses and N-body simulations and select the hierarchical triple
%systems. I calculate the waveform from the stellar-mass binary black holes and will study the existence and masses of
%IMBHs with the population properties of sources resolved by LISA/TianQin. For the sources undetectable, I plan to use
%them to study IMBHs by the multiband observation with LIGO/Virgo/KAGRA. This research will provide a novel approach to
%detecting IMBHs.

In the future (if I have the honor to join your group), I will shift my attention to detection for intermediate-mass binary black holes (IMBBHs). In the early
University, IMBHs could be formed via the evolution of Population III stars, and then formed IMBBHs. I will generate the IMBBHs by population syntheses and N-body simulations, and explore the capacity of space-borned and ground-based GW detectors for them. After that, I will study the potential to determine whether IMBHs resolved are formed
by this scenario depending on observation. I am also planning to study the extensions of general relativity (GR), because
IMBBHs could be good sources for testing GR. In addition, I plan to do some research on cosmology like constraining the
Hubble constant by IMBHHs. These research will
contribute to the study on the evolution of Population III stars, formation of IMBHs, extensions of GR, dynamical properties of clusters, as well as cosmology.

My current research on hierarchical triple black hole systems is relative to studying the extensions of GR, because the
impact of the tertiary on the GWs from the inner binaries may be degenerate with the modifications of GR. This research
may help break the degeneracy, contributing to the contraining the extensions of GR. In addition, I am willing to adjust my future research plan according to
your needs.

My research field is gravitational wave astronomy, focusing on studying stellar- and intermediate-mass black holes. I am
capable of generating sources by Monte Carlo simulation and population syntheses and N-body simulations. I also know
some data analysis, such Fisher information matrix and Bayesian inference. I have a good cooperative relationship with
experts in TianQin Research Center for Gravitational Physics Sun Yat-sen University, Peking University, and Caltech. I believe that if I am lucky enough to join your team, I can work well with you, benefit from each other, inspire chemistry between us, and enrich your scientific payoff. 


%{\hskip 2em}If I am lucky enought to join your team, we can have a good cooperative relationship. My current and future
%research will contribute to the study of LISA/LIGO on IMBHs and enrich their scientific payoff.  

%{\hskip 2em}I cultivate the following attitudes in students who complete course work or research with me:
%
%$\bullet{}$ demonstrate understanding of rigorous mathematical tools for design/analysis, (Teach others)

\end{document}
