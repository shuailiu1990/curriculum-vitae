%%%%%%%%%%%%%%%%%%%%%%%%%%%%%%%%%%%%%%%%%
% "ModernCV" CV and Cover Letter
% LaTeX Template
% Version 1.1 (9/12/12)
%
% This template has been downloaded from:
% http://www.LaTeXTemplates.com
%
% Original author:
% Xavier Danaux (xdanaux@gmail.com)
%
% License:
% CC BY-NC-SA 3.0 (http://creativecommons.org/licenses/by-nc-sa/3.0/)
%
% Important note:
% This template requires the moderncv.cls and .sty files to be in the same
% directory as this .tex file. These files provide the resume style and themes
% used for structuring the document.
% Statement on "Research Statement" https://careers.uw.edu/wp-content/uploads/sites/25/2016/06/Research-Statements.pdf
%%%%%%%%%%%%%%%%%%%%%%%%%%%%%%%%%%%%%%%%%

%----------------------------------------------------------------------------------------
%	PACKAGES AND OTHER DOCUMENT CONFIGURATIONS
%----------------------------------------------------------------------------------------

\documentclass[12pt,a4paper,sans]{article}%{moderncv} % Font sizes: 10, 11, or 12; paper sizes: a4paper, letterpaper, a5paper, legalpaper, executivepaper or landscape; font families: sans or roman
\usepackage{setspace} 
\setstretch{1.2}

%\moderncvstyle{classic} % CV theme - options include: 'casual' (default), 'classic', 'oldstyle' and 'banking'
%\moderncvcolor{grey} % CV color - options include: 'blue' (default), 'orange', 'green', 'red', 'purple', 'grey' and 'black'

\usepackage{lipsum} % Used for inserting dummy 'Lorem ipsum' text into the template
%\usepackage{caption}
%\captionsetup{font=10}
\usepackage[scale=0.75]{geometry} % Reduce document margins
%\setlength{\hintscolumnwidth}{3cm} % Uncomment to change the width of the dates column
%\setlength{\makecvtitlenamewidth}{10cm} % For the 'classic' style, uncomment to adjust the width of the space allocated to your name

%----------------------------------------------------------------------------------------
%	NAME AND CONTACT INFORMATION SECTION
%----------------------------------------------------------------------------------------
%\title{Research Statement}
%\maketitle
%\firstname{Shuai} % Your first name
%\familyname{Liu} % Your last name

% All information in this block is optional, comment out any lines you don't need
\title{\vspace{-2.5cm}\Huge Research Statement \vspace{-2.2em}}
\date{}
%\section{Research Statement}
%\address{W. Ethan Eagle}{}
%\mobile{(302) 584 3464}
%\phone{(000) 111 1112}
%\fax{(000) 111 1113}
%\email{}
%\homepage{staff.org.edu/~jsmith}{staff.org.edu/$\sim$jsmith} % The first argument is the url for the clickable link, the second argument is the url displayed in the template - this allows special characters to be displayed such as the tilde in this example
%\extrainfo{additional information}
%\photo[70pt][0.4pt]{pictures/picture} % The first bracket is the picture height, the second is the thickness of the frame around the picture (0pt for no frame)
%\quote{"A witty and playful quotation" - John Smith}

%----------------------------------------------------------------------------------------

\begin{document}
\maketitle
%\makecvtitle % Print the CV title

%----------------------------------------------------------------------------------------
My research interests focus on exploring promising sources for future gravitational wave detectors, for example, stellar- and
intermediate-mass black holes as well as massive black holes. Specially, my researches involve detection
and source parameters estimation for these black holes, constraining their population properties as well as studying
cosmology by simulated GW observation data. My research statement is organized as
follows:  

I worked on stellar-mass binary black holes (SBBHs), during my PhD study. SBBHs detected by
gravitational waves (GWs) are heavier than those observed by electromagnetic waves. These heavy SBBHs are promising sources for scheduled
space-borne GW detectors focusing on millihertz, e.g., LISA and TianQin. We adopted five mass
distribution models of SBBHs calibrated by events detected to study SBBHs with LISA and TianQin. We estimated the
strength of GW signals in data, and predicted that dozens of sources would be resolved. We predicted that source parameters
would be measured accurately by performing the Fisher information matrix on GW data, and the formation scenarios of sources
could be distinguished by measuring their orbital eccentricities. In addition, the detection and measurement capacities
for SBBHs could be improved significantly by joint
observation of TianQin and LISA. This work [published in Phys.Rev.D 101 (2020) 10, 103027] provides basis and guidance for future GW
data analysis and joint observation of detectors networks. 

I shift my attention to intermediate-mass black holes (IMBHs) [host China Postdoctoral Science Foundation, Grant No. 2021TQ0389 (180K CNY)], after I become a postdoctoral fellow. LIGO/Virgo observed the SBBH named
GW190521 whose merger remnant falls in the mass range of IMBHs. The event directly confirms the existence of
IMBHs and that IMBHs could be formed by mergers of SBBHs for the first time. Same as before, we simulated GW190521-like SBBHs in the
Universe, and analyzed GW signals from them in observation data. A dozen of sources could be detected and their source
parameters could be measured with high precision by LISA/TianQin. GW190521-like SBBHs with extremely large orbital
eccentricities would be identified by
multiband observation of LISA/TianQin and the third generation ground-based detectors Cosmic Explore (CE)/Einstein
Telescope (ET). We also inferred the Hubble constant with GW signals of LISA/TianQin by the Bayesian method, the precision could reach
~10\%. This work [published in Phys.Rev.D 105 (2022) 2, 023019] provides
basis and gauidence for searching GW190521-like SBBHs in observation data of future GW detectors and studying
cosmology by them. Meanwhile, I participated in constraining the Hubble
constant with the GW signals from SBBHs detected by multiband observation (e.g., LISA+ET) as one of the important
contributors. The Hubble constant would be constrained within ~ 1\% by the Bayesian method. This work [published in Sci.China Phys.Mech.Astron. 65 (2022) 5, 259811] is of great
importance for studying cosmology with SBBHs detected by future multiband observation. Now, I am working on
exploring IMBHs in Population III star clusters by furture detectors. IMBHs could be
formed by massive stars with low metallicity and form hierarchical triple black holes by capturing SBBHs. I plan to
study IMBHs by SBBHs around them detected by GWs in three aspects: distribution of orbital
eccentricities, evolution of orbital eccentricities, and multiband observation number. I have completed the evolution of
Population III star clusters by $N$-body simulation and selected triple systems. I am working on the detection for 
GW signals from SBBHs in triple systems. This work will provide a novel
approach to detect and study IMBHs with GW data available in the future.

In the future, I am going to investigate massive black holes (MBHs) with GW data of future detector networks. Electromagnetic
observations demonstrate that MBHs exist in the centers of most
galaxies in the local Universe, but their origin and evolution at high redshift remains largely
elusive, e.g., the mixture ratio between MBHs formed by light and heavy seed black holes. LISA/TianQin, as well as the
CE/ET could detect MBHs up to much higher
redshift (10-20). LISA/TianQin focusing on millihertz are sensitive to heavy seeds at high redshift, while CE/ET focusing
hundreds of hertz are sensitive to light seeds at low redshift, so detectors networks (e.g., LISA+CE, TianQin+ET) with different sensitive bands are expected to
constrain the formation and evolution of MBHs over a wide mass spectrum and across a more complete cosmic
epoch. In this project, I will construct an improved semi-analytical model of MBHs by combining
existing ones, in order to reduce bias in simulation for formation and evolution of MBHs first.
Secondly, I will analyze GW signals as same before, calculate detection number, and then estimate source parameters
with FIM and the
Baysesian method respectively. Finally, I plan to use the hierarchical Bayesian inference to constrain the mixture ratio
between MBHs formed by light and heavy seeds with GW data of detector networks. In addition, I will investigate the impact from different design configurations of
detectors, e.g., different arm lengths and noise of LISA, on the results. If it is slow to simulate the population of
MBHs with the improved
semi-analytical model and estimate source parameters as well as mixture ratio with the Bayesian method, we will
accelerate these processes by machine learning. This project will provide a new and complete framework to explore the
origin and evolution of MBHs with GW data of detector networks expected to be
available in the future. It will also it will provide important guidance for the final decision for the design
configuration of these forthcoming detectors. This project would be completed in the future, because of the knowledge
and skills I got from previous research.

I focus on studying black holes with GWs. I have rich knowledge in astrophysics, experience and skills (e.g., the Bayesian
inference and $N$-body simulation) in GW
data analysis. I believe that I could meet your needs in the area (Machine learning and Data Science for cosmology and astroparticle
physics, including gravitational wave astrophysics). If I have the honor to join your team, I can work well with you, benefiting from each
other and enriching your scientific payoff. 

%Intermediate-mass black holes (IMBHs) are very important for astronomy and physics, because they may shed
%light on the formation of massive black holes, the evolution of galaxies, as well as the effect of dark matter. However, IMBHs
%are still elusive, due to the lack of direct observation for them. In the case that the detection for IMBHs with
%electromagnetic waves is challenging, gravitational waves are expected to contribute significant help for studying them 
%as an entirely new probe. So, my research concentrates on the detection for IMBHs with gravitational waves.
%
%The IMBHs could be sought by observing GW190521-like binary black holes, due to the
%remnant of GW190521 with $\sim163M_{\odot}$. This kind of binary black holes would be in the sensitive bands of the
%future space-borne gravitational wave detectors, such as LISA and TianQin, and last for several years before they
%merge into LIGO/Virgo bands. So, in my past about one year as a postdoctoral researcher, I simulated this kind of binary
%black hole in the Universe depending on the masses and merger rate of GW190521. I studied the capacity for detection of GW190521-like
%binary black holes with LISA/TianQin and the potential to constrain their formation scenarios by measuring the orbital
%eccentricities. As the number of this kind of binary black holes detected increases, more accurate population models will
%be released. I will use the updated population model to obtain more accurate predictions of space gravitational wave
%detectors. IMBHs with more masses could be formed in globular
%clusters, and they would form a hierarchical triple system with stellar-mass binary black holes around them. The IMBHs
%will affect the evolution of orbital elements of stellar-mass binary black holes in these systems, making the
%gravitational waves from them carry the information on IMBHs. The stellar-mass binary black holes during the above phase
%are far away from the merger, they fall in the LISA/TianQin bands. So the hierarchical triple system could be used to
%detect the IMBHs in globular clusters, which is the research I am working on. Specially, I simulate the evolution of
%globular clusters hosting IMBHs by population syntheses and N-body simulations and select the hierarchical triple
%systems. I calculate the waveform from the stellar-mass binary black holes and will study the existence and masses of
%IMBHs with the population properties of sources resolved by LISA/TianQin. For the sources undetectable, I plan to use
%them to study IMBHs by the multiband observation with LIGO/Virgo/KAGRA. This research will provide a novel approach to
%detecting IMBHs.

%The IMBHs could be sought by observing GW190521-like binary black holes, due to the
%remnant of GW190521 with $\sim163M_{\odot}$. This kind of binary black holes would be in the sensitive bands of the
%future space-borne gravitational wave detectors, such as LISA and TianQin, and last for serveral years before they
%merge into LIGO/Virgo bands. So, in my past about one year as a postdoctoral researcher, I studied the capacity for detection of GW190521-like
%binary black holes with LISA/TianQin. I simulated this kind of binary black hole in the Universe depending on the
%masses and merger rate of GW190521, and then calculated the expected detection number and source parameter estimation
%precision by the Fisher information matrix. The results showed that LISA/TianQin would be able to tell LIGO/Virgo/KAGRA when and where they would merge to IMBHs in advance, depending on the high estimation precision on merger time and sky localization. In addition, the formation scenarios and sites of
%these sources could be constrained with the orbital eccentricities measurement. IMBHs with more masses could be formed in globular
%clusters, and they would form a hierarchical triple system with stellar-mass binary black holes around them. The IMBHs
%will affect the evolution of orbital elements of stellar-mass binary black holes in these systems, making the
%gravitational waves from them carry the information on IMBHs. The stellar-mass binary black holes during the above phase
%are far away from the merger, they fall in the LISA/TianQin bands. So the hierarchical triple system could be used to
%detect the IMBHs in globular clusters, which is the research I am working on. Specially, I simulate the evolution of
%globular clusters hosting IMBHs by population syntheses and N-body simulations and select the hierarchical triple
%systems. I calculate the waveform from the stellar-mass binary black holes and will study the existence and masses of
%IMBHs with the population properties of sources resolved by LISA/TianQin. For the sources undetectable, I plan to use
%them to study IMBHs by the multiband observation with LIGO/Virgo/KAGRA. This research will provide a novel approach to
%detecting IMBHs.

%In the future (if I have the honor to join your group), I will shift my attention to detection for intermediate-mass binary black holes (IMBBHs). In the early
%University, IMBHs could be formed via the evolution of Population III stars, and then formed IMBBHs. I will generate the IMBBHs by population syntheses and N-body simulations, and explore the capacity of space-borned and ground-based GW detectors for them. After that, I will study the potential to determine whether IMBHs resolved are formed
%by this scenario depending on observation. I am also planning to study the extensions of general relativity (GR), because
%IMBBHs could be good sources for testing GR. In addition, I plan to do some research on cosmology like constraining the
%Hubble constant by IMBHHs. These research will
%contribute to the study on the evolution of Population III stars, formation of IMBHs, extensions of GR, dynamical properties of clusters, as well as cosmology.

%My current research on hierarchical triple black hole systems is relative to studying the extensions of GR, because the
%impact of the tertiary on the GWs from the inner binaries may be degenerate with the modifications of GR. This research
%may help break the degeneracy, contributing to the contraining the extensions of GR. In addition, I am willing to adjust my future research plan according to
%your needs.



%{\hskip 2em}If I am lucky enought to join your team, we can have a good cooperative relationship. My current and future
%research will contribute to the study of LISA/LIGO on IMBHs and enrich their scientific payoff.  

%{\hskip 2em}I cultivate the following attitudes in students who complete course work or research with me:
%
%$\bullet{}$ demonstrate understanding of rigorous mathematical tools for design/analysis, (Teach others)

\end{document}
