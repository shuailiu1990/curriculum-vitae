%%%%%%%%%%%%%%%%%%%%%%%%%%%%%%%%%%%%%%%%%
% "ModernCV" CV and Cover Letter
% LaTeX Template
% Version 1.1 (9/12/12)
%
% This template has been downloaded from:
% http://www.LaTeXTemplates.com
%
% Original author:
% Xavier Danaux (xdanaux@gmail.com)
%
% License:
% CC BY-NC-SA 3.0 (http://creativecommons.org/licenses/by-nc-sa/3.0/)
%
% Important note:
% This template requires the moderncv.cls and .sty files to be in the same
% directory as this .tex file. These files provide the resume style and themes
% used for structuring the document.
% Statement on "Research Statement" https://careers.uw.edu/wp-content/uploads/sites/25/2016/06/Research-Statements.pdf
%%%%%%%%%%%%%%%%%%%%%%%%%%%%%%%%%%%%%%%%%

%----------------------------------------------------------------------------------------
%	PACKAGES AND OTHER DOCUMENT CONFIGURATIONS
%----------------------------------------------------------------------------------------

\documentclass[12pt,a4paper,sans]{article}%{moderncv} % Font sizes: 10, 11, or 12; paper sizes: a4paper, letterpaper, a5paper, legalpaper, executivepaper or landscape; font families: sans or roman
\usepackage{setspace}
\usepackage{CJKutf8}
\setstretch{1.5}

%\moderncvstyle{classic} % CV theme - options include: 'casual' (default), 'classic', 'oldstyle' and 'banking'
%\moderncvcolor{grey} % CV color - options include: 'blue' (default), 'orange', 'green', 'red', 'purple', 'grey' and 'black'

\usepackage{lipsum} % Used for inserting dummy 'Lorem ipsum' text into the template
%\usepackage{caption}
%\captionsetup{font=10}
\usepackage[scale=0.75]{geometry} % Reduce document margins
%\setlength{\hintscolumnwidth}{3cm} % Uncomment to change the width of the dates column
%\setlength{\makecvtitlenamewidth}{10cm} % For the 'classic' style, uncomment to adjust the width of the space allocated to your name

%----------------------------------------------------------------------------------------
%	NAME AND CONTACT INFORMATION SECTION
%----------------------------------------------------------------------------------------
%\title{Research Statement}
%\maketitle
%\firstname{Shuai} % Your first name
%\familyname{Liu} % Your last name

% All information in this block is optional, comment out any lines you don't need
\begin{CJK}{UTF8}{gbsn}
\title{\vspace{-2.5cm}\Huge 研究经历综述 \vspace{-2.2em}}
\end{CJK}
\date{}
%\section{Research Statement}
%\address{W. Ethan Eagle}{}
%\mobile{(302) 584 3464}
%\phone{(000) 111 1112}
%\fax{(000) 111 1113}
%\email{}
%\homepage{staff.org.edu/~jsmith}{staff.org.edu/$\sim$jsmith} % The first argument is the url for the clickable link, the second argument is the url displayed in the template - this allows special characters to be displayed such as the tilde in this example
%\extrainfo{additional information}
%\photo[70pt][0.4pt]{pictures/picture} % The first bracket is the picture height, the second is the thickness of the frame around the picture (0pt for no frame)
%\quote{"A witty and playful quotation" - John Smith}

%----------------------------------------------------------------------------------------

\begin{document}
\begin{CJK}{UTF8}{gbsn}
\maketitle
%\makecvtitle % Print the CV title
\CJKindent本人自攻读博士以来一直从事未来引力波探测器引力波源的研究,特别是恒星级与中等质量黑洞。研究主要涉及评估探测器的探测能力,比如探测数、源参数测量估计精度,以及限制源的族群性质与形成机制。在硕士期间,从事利用全息方法或者反德西特与全息场论之间的对应(AdS/CFT)研究超导的性质,主要研究对象为约瑟夫森节。下面我将对自博士期间以来的研究工作进行介绍。

在我攻读博士期间,从事利用未来空间引力波探测器研究恒星级双黑洞的相关工作。人们发现通过地面引力波探测器LIGO与Virgo探测到的恒星级黑洞要比通过电磁波观测到的重许多,因此这些较重的恒星级双黑洞有可能被未来的LISA或天琴等空间引力波探测器探测到。我们首次采用了通过LIGO与Virgo观测到的恒星级双黑洞构造的恒星级双黑洞质量分布函数模拟宇宙中的恒星级双黑洞,并且利用LISA与天琴研究它们。我们通过计算发现未来会有几十个双黑洞被探测到,并且其质量等源参数能够被精确测量。同时发现源的形成机制可以通过测量其轨道离心率进行区分。此工作(发表在Phys.Rev.D101 (2020) 10, 103027,本人一作)为未来引力波探测器探测恒星级双黑洞与限制其形成机制提供了基础。

LIGO与Virgo观测到了一例特殊的恒星级双黑洞并合事件GW190521。该事件中质量较大的黑洞的质量处于质量间隙中(恒星演化理论预言该间隙中不存在恒星级黑洞)并且该事件并合后的黑洞处于中等质量黑洞的质量范围内。该事件首次在观测上分别直接证实了处于质量间隙中黑洞以及中等质量黑洞的存在。在成为博士后之后,我们利用引力波研究了此类特殊的双黑洞(以此研究主持中国博士后基金会站前特别资助,18万元人民币)。我们利用GW190521的质量与并合率模拟出宇宙中的类GW190521双黑洞,在此基础上评估了LISA与天琴对其探测能力。计算结果表明数个至十几个源会被探测到,并且其源参数会被精确测量。此外,我们发现当该类源的轨道离心率不是特别大时,可以通过测量其轨道离心率限制其形成机制。如果轨道离心率极大,可以通过LISA或天琴与爱因斯坦望远镜等第三代地面引力波探测器的多波段观测对其进行探测以及轨道离心率的限制。该工作(发表在Phys.Rev.D105 (2022) 2, 023019,本人一作)提供了未来引力波探测器探测类GW190521双黑洞以及限制其形成机制的基础。同时,我作为重要参与者之一参与了利用恒星级双黑洞的多波段观测限制宇宙哈勃常数的研究。我们利用贝叶斯推断方法研究发现哈勃常数相对精度能够被限制到大约1\%。该工作(发表在Sci.China Phys.Mech.Astron. 65 (2022) 5, 259811, 本人为共同通讯作者之一)对利用未来多波段引力波探测研究宇宙学提供了重要参考意义。虽然LIGO与Virgo已经探测到了大约90例恒星级双黑洞并合事件,并且认为其中一些双黑洞的轨道离心率比较大。双黑洞处于三体系统是造成离心率较大猜测之一。由于引力波辐射的圆化作用,处于LIGO与Virgo频段的双黑洞的轨道离心率很小,因此无法对其所处环境(是否处于三体系统中)进行有效地推断。基于以上问题,我现在正在开展利用未来空间引力波探测星族III星团中由黑洞组成的等级三体系统(以下简称三体系统)的相关研究(恒星级双黑洞在LISA与天琴频段还未被引力波辐射圆化)。三体系统由恒星级双黑洞围绕第三个黑洞组成的系统。由于第三个黑洞的引力扰动对其周围的恒星级双黑洞轨道演化的影响,其演化模式会与孤立双黑洞系统不同。因此,我打算通过以下两个方面研究三体系统中的恒星级双黑洞:轨道离心率的分布以及轨道离心率的演化。然后,根据以上结果限制恒星级双黑洞的形成机制,即推断源是否处于三体系统中。目前,我已经利用多体模拟以及星族合成完成星族III星团的演化,以及三体系统的分析。该工作(arXiv:2311.05393)已经投到著名天文学期刊MNRAS上,目前正根据审稿意见进行修改。

球状星团被认为是中等质量黑洞最有可能所处的场所,但是由于电磁波观测手段的局限,无法对其中的中等质量黑洞进行有效观测。因此,我未来将开展球状星团中中等质量黑洞的引力波探测研究。由于球状星团中致密天体的密度较高,因此如果星团中存在中等质量黑洞,它洞预期会与其周围的恒星级双黑洞形成三体系统,并且通过引力影响后者的轨道演化,比如轨道离心率,从而使后者辐射的引力波携带中等质量黑洞的信息。我打算利用多体模拟与星族合成程序模拟具有中等质量黑洞的球状星团的演化,并且筛选出所要研究的三体系统。然后,分别研究三体系统中的恒星级双黑洞的轨道离心率的分布以及演化。注意,其中一些三体系统通过Kozi-Lidov效应使其中的双黑洞系统具有极高的轨道离心率,因此打算通过地面联合空间引力波探测器对这类双黑洞进行多波段观测。接下来,通过分析双黑洞的轨道离心率的分布以及演化推断中等质量黑洞的存在,并且研究质量等性质。这项研究将提供一种新颖的研究中等质量黑洞的方法。

以上为本人利用未来空间引力波探测恒星级与中等质量黑洞的相关研究。目前与未来的研究课题与贵校(广州大学)中的天文学方向联系紧密。如果我能有机会在贵校工作,相信能够为贵校发展,特别是天文学方向作出相应的贡献。
%----------------------------------------------------------------------------------------




%{\hskip 2em}If I am lucky enought to join your team, we can have a good cooperative relationship. My current and future
%research will contribute to the study of LISA/LIGO on IMBHs and enrich their scientific payoff.  

%{\hskip 2em}I cultivate the following attitudes in students who complete course work or research with me:
%
%$\bullet{}$ demonstrate understanding of rigorous mathematical tools for design/analysis, (Teach others)
\end{CJK}
\end{document}
