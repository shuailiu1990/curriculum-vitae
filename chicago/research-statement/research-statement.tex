%%%%%%%%%%%%%%%%%%%%%%%%%%%%%%%%%%%%%%%%%
% "ModernCV" CV and Cover Letter
% LaTeX Template
% Version 1.1 (9/12/12)
%
% This template has been downloaded from:
% http://www.LaTeXTemplates.com
%
% Original author:
% Xavier Danaux (xdanaux@gmail.com)
%
% License:
% CC BY-NC-SA 3.0 (http://creativecommons.org/licenses/by-nc-sa/3.0/)
%
% Important note:
% This template requires the moderncv.cls and .sty files to be in the same
% directory as this .tex file. These files provide the resume style and themes
% used for structuring the document.
% Statement on "Research Statement" https://careers.uw.edu/wp-content/uploads/sites/25/2016/06/Research-Statements.pdf
%%%%%%%%%%%%%%%%%%%%%%%%%%%%%%%%%%%%%%%%%

%----------------------------------------------------------------------------------------
%	PACKAGES AND OTHER DOCUMENT CONFIGURATIONS
%----------------------------------------------------------------------------------------

\documentclass[12pt,a4paper,sans]{article}%{moderncv} % Font sizes: 10, 11, or 12; paper sizes: a4paper, letterpaper, a5paper, legalpaper, executivepaper or landscape; font families: sans or roman
\usepackage{setspace} 
\setstretch{1.2}

%\moderncvstyle{classic} % CV theme - options include: 'casual' (default), 'classic', 'oldstyle' and 'banking'
%\moderncvcolor{grey} % CV color - options include: 'blue' (default), 'orange', 'green', 'red', 'purple', 'grey' and 'black'

\usepackage{lipsum} % Used for inserting dummy 'Lorem ipsum' text into the template
%\usepackage{caption}
%\captionsetup{font=10}
\usepackage[scale=0.75]{geometry} % Reduce document margins
%\setlength{\hintscolumnwidth}{3cm} % Uncomment to change the width of the dates column
%\setlength{\makecvtitlenamewidth}{10cm} % For the 'classic' style, uncomment to adjust the width of the space allocated to your name

%----------------------------------------------------------------------------------------
%	NAME AND CONTACT INFORMATION SECTION
%----------------------------------------------------------------------------------------
%\title{Research Statement}
%\maketitle
%\firstname{Shuai} % Your first name
%\familyname{Liu} % Your last name

% All information in this block is optional, comment out any lines you don't need
\title{\vspace{-2.5cm}\Huge Research Statement \vspace{-2.2em}}
\date{}
%\section{Research Statement}
%\address{W. Ethan Eagle}{}
%\mobile{(302) 584 3464}
%\phone{(000) 111 1112}
%\fax{(000) 111 1113}
%\email{}
%\homepage{staff.org.edu/~jsmith}{staff.org.edu/$\sim$jsmith} % The first argument is the url for the clickable link, the second argument is the url displayed in the template - this allows special characters to be displayed such as the tilde in this example
%\extrainfo{additional information}
%\photo[70pt][0.4pt]{pictures/picture} % The first bracket is the picture height, the second is the thickness of the frame around the picture (0pt for no frame)
%\quote{"A witty and playful quotation" - John Smith}

%----------------------------------------------------------------------------------------

\begin{document}
\maketitle
%\makecvtitle % Print the CV title

%----------------------------------------------------------------------------------------
I focus on exploring binary black holes with future gravitational wave detectors, which includes stellar- and
intermediate-mass black holes. It involves primarily assessing detection capabilities (detection number and source
parameters estimation), as well as constraining population properties and formation channels of sources. My research statement is organized as
follows:  

I worked on stellar-mass binary black holes (SBBHs), during my PhD study. SBBHs detected by
gravitational waves (GWs) are heavier than those observed by electromagnetic waves. These heavy SBBHs are promising sources for scheduled
space-borne GW detectors focusing on millihertz, e.g., LISA and TianQin. We adopted five mass
distribution models of SBBHs calibrated by events detected LIGO/Virgo to study them with LISA and TianQin. We predict that dozens of sources would be resolved, and source parameters
would be measured accurately. We also argue that formation scenarios (isolated binary evolution and dynamical interaction) of sources
could be distinguished by measuring their orbital eccentricities. In addition, the detection and measurement
capabilities 
for SBBHs could be improved significantly by joint
observation of TianQin and LISA. This work [published in Phys.Rev.D 101 (2020) 10, 103027] provides a basis and guidance for detecting SBBHs and distinguishing their formation channels by future GW
detectors. 

LIGO/Virgo observed a special SBBH event named
GW190521 whose primary component mass and merger remnant fall in the mass gap and mass range of intermediate-mass black holes (IMBHs), respectively. The event directly confirms the existence of
black holes in the mass gap and IMBHs for the first time. We investigate GW190521-like sources [host China Postdoctoral Science Foundation, Grant No. 2021TQ0389 (180K CNY)], after I become a postdoctoral fellow. Same as before, we simulate GW190521-like SBBHs in the
Universe by the mass and merger rate of GW190521, and estimate the detection capabilities of LISA/TianQin for them. A dozen of sources could be detected and their source
parameters could be measured with high precision. We could tell whether sources originate from isolated binary evolution or dynamical interaction by measuring their orbital eccentricities. For sources with extremely large orbital
eccentricities, they would be identified by multiband observation of LISA (TianQin) and the third generation
ground-based detectors Cosmic Explorer (CE) or Einstein
Telescope (ET). This work [published in Phys.Rev.D 105 (2022) 2, 023019] provides
the basis and guidance for detecting GW190521-like sources and studying their formation channels by future GW detectors. Meanwhile, I participated in constraining the Hubble
constant with SBBHs detected by multiband observation (e.g., LISA+ET) as one of the important
contributors. The Hubble constant would be constrained within ~ 1\% by the Bayesian method. This work [published in Sci.China Phys.Mech.Astron. 65 (2022) 5, 259811] is of great
importance for studying cosmology with SBBHs detected by future multiband GW observation. Now, I am working on
exploring inner SBBHs in hierarchical triple black hole systems in Population III star clusters by future GW detectors. The orbital evolution of the SBBH could be affected by the third black hole, so I plan to
study these SBBHs by GWs in three different aspects: distribution of orbital
eccentricities, the evolution of orbital eccentricities, as well as multiband observation numbers. Then, I will constrain the formation scenarios of SBBHs depending on these results. I have completed the evolution of
Population III star clusters by $N$-body simulation and selected hierarchical triple black hole systems. I am analyzing
their population properties. This work will provide a method 
to study population properties and formation mechanisms of SBBHs in hierarchical triple black hole systems in Population III star clusters by forthcoming GW detectors.

In the future, I will shift my attention to hierarchical triple black hole systems in globular clusters (GCs) where
IMBHs are generally thought to exist. Due to the dense environment of GCs, SBBHs and IMBHs would form hierarchical
triple black hole systems, if there exist IMBHs. The orbital evolution of SBBHs would be affected by IMBHs, making GWs
from the former carry informaiton of the latter. Same as before, I will evolve GCs harboring IMBHs with $N$-body simulation and
select the targeted hierarchical triple systems formed. I will study SBBHs in triple systems by future GW detections in the distribution and evolution of orbital
eccentricities, as well as multiband observation number, respectively. Then, I will analyze the population properties of these SBBHs and infer the existence of IMBHs depending on the above results. This project would provide a novel approach to detect IMBHs and explore the formation channels of SBBHs in GCs by future GW detections.  

I focus on studying SBBHs and IMBHs with GWs. I think that my current and future and research projects on
hierarchical triple
black holes systems are very relative to one of current topics of research in the institute, i.e., gravitational wave studies. I believe that my current and future research will be completed, due to my rich knowledge in astrophysics, experience and skills (e.g., gravitational wave and $N$-body
simulation). If I have an opportunity to be awarded the fellowship, I think we could benifit and enrich scientific payoff from each other. 



%{\hskip 2em}If I am lucky enought to join your team, we can have a good cooperative relationship. My current and future
%research will contribute to the study of LISA/LIGO on IMBHs and enrich their scientific payoff.  

%{\hskip 2em}I cultivate the following attitudes in students who complete course work or research with me:
%
%$\bullet{}$ demonstrate understanding of rigorous mathematical tools for design/analysis, (Teach others)

\end{document}
