%!TEX encoding = UTF8
%!TEX root =cv-llt.tex

\begin{rubric}{会议报告}
	\entry*[2024] \textbf{肇庆市第一中学科普报告},引力波之简介
    \entry*[2024] \textbf{第一届音频波段引力波天文学研讨会报告},星族III星团中中等质量黑洞的引力波探测
	\entry*[2023] \textbf{主办中山大学-北京大学引力波天文学研讨会}
	\entry*[2023] \textbf{海德堡大学在线邀请报告},利用空间引力波探测器研究恒星级与中等质量黑洞
	\entry*[2023] \textbf{北京大学科维理天文与天体物理研究所在线邀请报告},利用空间引力波探测器研究恒星级与中等质量黑洞
	\entry*[2022] \textbf{中国物理学会引力与相对论天体物理学分会2022年分会报告},利用空间引力波探测器研究恒星级双黑洞
	\entry*[2022] \textbf{引力波未来会议在线邀请报告},利用空间引力波探测器研究恒星级双黑洞
	\entry*[2022] \textbf{南方科技大学邀请报告},利用天琴探测恒星级双黑洞
	\entry*[2021] \textbf{重庆大学与中国科学技术大学青年座谈会报告},天琴科学目标:恒星级双黑洞
	\entry*[2021] \textbf{中英联合引力波卓越联合培训联盟报告},天琴科学目标:恒星级双黑洞
	\entry*[2020] \textbf{第13届京广夏天体物理研讨会报告},天琴科学目标:恒星级双黑洞
	\entry*[2020] \textbf{第7届KAGRA国际在线研讨会报告},天琴科学目标:恒星级双黑洞
	\entry*[2019] \textbf{中山大学天琴中心与华中科技大学引力中心年会报告},天琴对恒星级双黑洞探测能力的研究
	\entry*[2018] \textbf{中山大学天琴中心与华中科技大学引力中心年会海报},天琴对恒星级双黑洞探测能力的研究
\end{rubric}
