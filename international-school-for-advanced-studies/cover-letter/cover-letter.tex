%%%%%%%%%%%%%%%%%%%%%%%%%%%%%%%%%%%%%%%%%
% Plain Cover Letter
% LaTeX Template
% Version 1.0 (28/5/13)
%
% This template has been downloaded from:
% http://www.LaTeXTemplates.com
%
% Original author:
% Rensselaer Polytechnic Institute 
% http://www.rpi.edu/dept/arc/training/latex/resumes/
%
% License:
% CC BY-NC-SA 3.0 (http://creativecommons.org/licenses/by-nc-sa/3.0/)
%
%%%%%%%%%%%%%%%%%%%%%%%%%%%%%%%%%%%%%%%%%

%----------------------------------------------------------------------------------------
%	PACKAGES AND OTHER DOCUMENT CONFIGURATIONS
%----------------------------------------------------------------------------------------

\documentclass[11pt,a4paper,sans]{letter} % Default font size of the document, change to 10pt to fit more text
\usepackage{geometry}
\geometry{top=6.0cm, bottom=0cm}
%\usepackage{newcent} % Default font is the New Century Schoolbook PostScript font 
%\usepackage{helvet} % Uncomment this (while commenting the above line) to use the Helvetica font

% Margins
\topmargin=-1in % Moves the top of the document 1 inch above the default
\textheight=10in%8.5in % Total height of the text on the page before text goes on to the next page, this can be increased in a longer letter
\oddsidemargin=-5pt % -10pt % Position of the left margin, can be negative or positive if you want more or less room
\textwidth=6.5in % Total width of the text, increase this if the left margin was decreased and vice-versa
\date{}
\let\raggedleft\raggedright % Pushes the date (at the top) to the left, comment this line to have the date on the right

\begin{document}

%----------------------------------------------------------------------------------------
%	ADDRESSEE SECTION
%----------------------------------------------------------------------------------------

\begin{letter}{September 20 2022 \\
%Prof. Alvin Chua and Prof. Yanbei Chen \\
SISSA\\
Trieste\\ 
Friuli-Venezia Giulia 34136, Italy}

Shuai Liu \\
TianQin Research Center for Gravitational Physics \\
Zhuhai Campus Sun Yat-sen University \\
Tangjiawan, Zhuhai 519082 \\
Guangdong, People's Republic of China \\
shuai.liu.1990@outlook.com




%----------------------------------------------------------------------------------------
%	YOUR NAME & ADDRESS SECTION
%----------------------------------------------------------------------------------------

%\begin{center}
%\large\bf Dr. John Smith \\ % Your name
%%\vspace{20pt} \hrule height 1pt % If you would like a horizontal line separating the name from the address, uncomment the line to the left of this text
%123 Broadway \\ City, State 12345 \\ (000) 111-1111 % Your address and phone number
%\end{center} 
%\vfill

%\signature{John Smith} % Your name for the signature at the bottom

%----------------------------------------------------------------------------------------
%	LETTER CONTENT SECTION
%----------------------------------------------------------------------------------------

\opening{To whom it may concern,} 
 
    I am writing to express my sincere interest in postdoctoral positions in the area (Machine learning and Data Science
    for cosmology and astroparticle physics, including gravitational wave astrophysics) at SISSA. I would love to
pursue my postdoctoral studies as a member of your team. I am a postdoctoral fellow, who is working on black holes with
    gravitational waves at Sun Yat-sen University.

I have been working on exploring binary black holes with gravitational waves, since I was a PhD candidate. I
    have rich astrophysics knowledge on binary black holes, and experience as well as technical skills in gravitational wave data analysis. During my PhD study, I focused on exploring stellar-mass binary black holes by future
    space-borne gravitational wave detectors (e.g., LISA and TianQin). Specifically, we estimated the strength of
    gravitational wave signals in observation data and detection number, and then estimated
    measurement precision for source parameters with the Fisher information matrix. During my postdoctoral stage, I shifted
    my attention to intermediate-mass black holes. Besides studying GW190521-like binary black holes whose merger remnants are
    intermediate-mass black holes as we explored stellar-mass binary black holes before, we constrained the
    Hubble constant with gravitational wave signals from them by the Bayesian method. Meanwhile, I also participated
    in the project which inferred the Hubble
    constant with gravitational wave signals from stellar-mass binary black holes as one of the most important
    contributors. Now, I am working on
    hierarchical triple
    black hole systems whose perturbers are intermediate-mass black holes in Population III star clusters. I plan to
    study intermediate-mass black holes by detecting gravitational wave signals from stellar-mass binary black holes
    around them. I have completed simulation for Population III
    clusters by $N$-body codes and selected triple systems.  
    In the future, I plan to explore the origin and evolution of massive binary black holes with gravitational wave
    data of future detector networks(e.g., LISA+TianQin) by the hierarchical Bayesian inference.

I would appreciate the opportunity to work in your group at SISSA. I am a self-motivated, independent researcher with
    passions for science as well data analysis. I believe that my knowledge,
    technical skills, and experience (e.g., the Bayesian method and $N$-body simulation) on studying black holes by
    gravitational waves could meet your expectations and aid in your research. I have attached my CV and would be very interested in setting up a time further
discuss my skills and qualifications with you. Please let me know whether you have questions and I look forward to
hearing from you.\\

Sincerely,

Shuai Liu


%\encl{Curriculum vitae, employment form} % List your enclosed documents here, comment this out to get rid of the "encl:"

%----------------------------------------------------------------------------------------

\end{letter}

\end{document}
