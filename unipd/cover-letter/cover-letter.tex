%%%%%%%%%%%%%%%%%%%%%%%%%%%%%%%%%%%%%%%%%
% Plain Cover Letter
% LaTeX Template
% Version 1.0 (28/5/13)
%
% This template has been downloaded from:
% http://www.LaTeXTemplates.com
%
% Original author:
% Rensselaer Polytechnic Institute 
% http://www.rpi.edu/dept/arc/training/latex/resumes/
%
% License:
% CC BY-NC-SA 3.0 (http://creativecommons.org/licenses/by-nc-sa/3.0/)
%
%%%%%%%%%%%%%%%%%%%%%%%%%%%%%%%%%%%%%%%%%

%----------------------------------------------------------------------------------------
%	PACKAGES AND OTHER DOCUMENT CONFIGURATIONS
%----------------------------------------------------------------------------------------

\documentclass[11pt,a4paper,sans]{letter} % Default font size of the document, change to 10pt to fit more text
\usepackage{geometry}
\usepackage{enumitem}
\geometry{top=6.0cm, bottom=0cm}
%\usepackage{newcent} % Default font is the New Century Schoolbook PostScript font 
%\usepackage{helvet} % Uncomment this (while commenting the above line) to use the Helvetica font

% Margins
\topmargin=-1.2in % Moves the top of the document 1 inch above the default
\textheight=10in%8.5in % Total height of the text on the page before text goes on to the next page, this can be increased in a longer letter
\oddsidemargin=-5pt % -10pt % Position of the left margin, can be negative or positive if you want more or less room
\textwidth=6.5in % Total width of the text, increase this if the left margin was decreased and vice-versa
\date{}
\let\raggedleft\raggedright % Pushes the date (at the top) to the left, comment this line to have the date on the right

\begin{document}

%----------------------------------------------------------------------------------------
%	ADDRESSEE SECTION
%----------------------------------------------------------------------------------------

\begin{letter}{September 20 2022 \\
%Prof. Alvin Chua and Prof. Yanbei Chen \\
University of Padua\\
Padua\\ 
Veneto 35122, Italy}

Shuai Liu \\
TianQin Research Center for Gravitational Physics \\
Zhuhai Campus Sun Yat-sen University \\
Tangjiawan, Zhuhai 519082 \\
Guangdong, People's Republic of China \\
shuai.liu.1990@outlook.com




%----------------------------------------------------------------------------------------
%	YOUR NAME & ADDRESS SECTION
%----------------------------------------------------------------------------------------

%\begin{center}
%\large\bf Dr. John Smith \\ % Your name
%%\vspace{20pt} \hrule height 1pt % If you would like a horizontal line separating the name from the address, uncomment the line to the left of this text
%123 Broadway \\ City, State 12345 \\ (000) 111-1111 % Your address and phone number
%\end{center} 
%\vfill

%\signature{John Smith} % Your name for the signature at the bottom

%----------------------------------------------------------------------------------------
%	LETTER CONTENT SECTION
%----------------------------------------------------------------------------------------

\opening{To Dear Prof. Michela Mapelli,} 
 
    I am writing to express my sincere interest in postdoctoral research positions founded by the ERC Consolidator Grant
    DEMOBLACK. I would love to
pursue my postdoctoral studies as a member of your team. I am a postdoctoral fellow, who is working on stellar-origin
    binary black holes with
    gravitational waves at Sun Yat-sen University.

I have been working on exploring stellar-origin binary black holes with gravitational waves, since I was a PhD
    candidate. During my PhD study, I focused on exploring stellar-origin binary black holes by future
    space-borne gravitational wave detectors (e.g., LISA and TianQin). Specifically, we estimated the detection capacity of
    these detectors, e.g., detection number and measurement precision for source parameters. We also explored the
    potential to distinguish the formation channels by measuring the orbital eccentricities. During my postdoctoral
    stage, I studied the GW190521-like stellar-mass binary black holes whose primary mass and merger remnant are in the
    mass gap and mass range of 
    intermediate-mass black holes. We explored this kind of stellar-mass binary black holes as before, including
    investigation for the potential to distinguish the formation channels by measuring the orbital eccentricities. 
    %Meanwhile, I participated
    %in a project which inferred the Hubble
    %constant with gravitational wave signals from stellar-origin binary black holes as one of the most important
    %contributors. 
    Now, I am working on
    hierarchical triple
    black hole systems whose inner binaries are stellar-origin binary black holes in Population III star clusters. I plan to
    study the population properties of these binary black holes by gravitation wave detections, e.g., distributions of
    mass and orbital eccentricity, and
    infer the formation channels depending on them. I have completed the simulation for Population III
    clusters by $N$-body codes and selected target hierarchical triple systems.  
    In the future, I am going to simulate the evolution of globular clusters with $N$-body codes, and then study the
    population properties of stellar-mass
    binary black holes in triple black holes and 
    investigate their formation channels by future gravitational wave detections.  

I would appreciate the opportunity to work in your group at University of Padua. I am a self-motivated, independent researcher with
    passions for stellar-origin binary black holes. I believe that my knowledge,
    technical skills and experience (e.g., $N$-body simulation) in studying stellar-origin binary black holes by
    gravitational waves could meet your expectations and aid in your research. I have attached my CV and would be very interested in setting up a time further
discuss my skills and qualifications with you. If you have questions, please let me know. I look forward to
hearing from you. 

Two recommendation letters from \begin{enumerate}[itemsep=-0pt,leftmargin=1.2em,topsep=-10pt]
\item Prof. Yi-Ming Hu, TianQin center for gravitational physics \& school of physics and astronomy, Sun Yat-sen University,
Guangdong province, China. $huyiming@mail.sysu.edu.cn$
\item Associate Prof. Long Wang, school of physics and astronomy, Sun Yat-sen University, Guangdong province, China.
$wanglong8@mail.sysu.edu.cn$\\
\end{enumerate}


         
Sincerely,

Shuai Liu


%\encl{Curriculum vitae, employment form} % List your enclosed documents here, comment this out to get rid of the "encl:"

%----------------------------------------------------------------------------------------

\end{letter}

\end{document}
